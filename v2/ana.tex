\section{ \reaction invariant mass analysis}
\subsection{overview}


\subsection{Event Selection}
The following event selection by the CDS is made prior to the beam line analysis to reduce CPU time.
\begin{itemize}
\item \# of CDH segments fired == 3
\item \# of good CDC tracks == 2
\end{itemize}
Here, clustering of CDH hits are not being made.\\
\begin{figure}
\includegraphics[width=\linewidth]{fig/hist33.pdf}
\caption{Hit multiplicity of CDH}
\end{figure}






\subsection{Beam Analysis}
\subsubsection{Analysis Cuts}

\begin{figure}
\includegraphics[width=\linewidth]{fig/hist2.pdf}
\caption{Hit multiplicity of BHD}
\end{figure}
\begin{figure}
\includegraphics[width=\linewidth]{fig/hist3.pdf}
\caption{Hit multiplicity of T0}
\end{figure}


\begin{figure}
\includegraphics[width=\linewidth]{fig/hist4.pdf}
\caption{Phase Space distribution of $\Sigma^+$ mode (left) and $\Sigma^-$ mode (right).}
\end{figure}

\begin{figure}
\includegraphics[width=\linewidth]{fig/hist5.pdf}
\caption{Phase Space distribution of $\Sigma^+$ mode (left) and $\Sigma^-$ mode (right).}
\end{figure}


\begin{figure}
\includegraphics[width=\linewidth]{fig/hist6.pdf}
\caption{Phase Space distribution of $\Sigma^+$ mode (left) and $\Sigma^-$ mode (right).}
\end{figure}

\begin{figure}
\includegraphics[width=\linewidth]{fig/hist7.pdf}
\caption{Phase Space distribution of $\Sigma^+$ mode (left) and $\Sigma^-$ mode (right).}
\end{figure}

\begin{figure}
\includegraphics[width=\linewidth]{fig/hist8.pdf}
\caption{Phase Space distribution of $\Sigma^+$ mode (left) and $\Sigma^-$ mode (right).}
\end{figure}

\begin{figure}
\includegraphics[width=\linewidth]{fig/hist9.pdf}
\caption{Phase Space distribution of $\Sigma^+$ mode (left) and $\Sigma^-$ mode (right).}
\end{figure}

\begin{figure}
\includegraphics[width=\linewidth]{fig/hist10.pdf}
\caption{Phase Space distribution of $\Sigma^+$ mode (left) and $\Sigma^-$ mode (right).}
\end{figure}

\begin{figure}
\includegraphics[width=\linewidth]{fig/hist11.pdf}
\caption{Phase Space distribution of $\Sigma^+$ mode (left) and $\Sigma^-$ mode (right).}
\end{figure}

\begin{figure}
\includegraphics[width=\linewidth]{fig/hist12.pdf}
\caption{Phase Space distribution of $\Sigma^+$ mode (left) and $\Sigma^-$ mode (right).}
\end{figure}

\begin{figure}
\includegraphics[width=\linewidth]{fig/hist13.pdf}
\caption{Phase Space distribution of $\Sigma^+$ mode (left) and $\Sigma^-$ mode (right).}
\end{figure}

\begin{figure}
\includegraphics[width=\linewidth]{fig/hist14.pdf}
\caption{Phase Space distribution of $\Sigma^+$ mode (left) and $\Sigma^-$ mode (right).}
\end{figure}

\begin{figure}
\includegraphics[width=\linewidth]{fig/hist15.pdf}
\caption{Phase Space distribution of $\Sigma^+$ mode (left) and $\Sigma^-$ mode (right).}
\end{figure}

\begin{figure}
\includegraphics[width=\linewidth]{fig/hist16.pdf}
\caption{Phase Space distribution of $\Sigma^+$ mode (left) and $\Sigma^-$ mode (right).}
\end{figure}

\begin{figure}
\includegraphics[width=\linewidth]{fig/hist17.pdf}
\caption{Phase Space distribution of $\Sigma^+$ mode (left) and $\Sigma^-$ mode (right).}
\end{figure}

\begin{figure}
\includegraphics[width=\linewidth]{fig/hist18.pdf}
\caption{Phase Space distribution of $\Sigma^+$ mode (left) and $\Sigma^-$ mode (right).}
\end{figure}

\begin{figure}
\includegraphics[width=\linewidth]{fig/hist19.pdf}
\caption{Phase Space distribution of $\Sigma^+$ mode (left) and $\Sigma^-$ mode (right).}
\end{figure}

\begin{figure}
\includegraphics[width=\linewidth]{fig/hist20.pdf}
\caption{Phase Space distribution of $\Sigma^+$ mode (left) and $\Sigma^-$ mode (right).}
\end{figure}

\begin{figure}
\includegraphics[width=\linewidth]{fig/hist21.pdf}
\caption{Phase Space distribution of $\Sigma^+$ mode (left) and $\Sigma^-$ mode (right).}
\end{figure}



\subsubsection{$K^-$ Beam Luminosity}



\subsection{$\pi^{+}$ and $\pi^{-}$ identification by the CDS}


\subsubsection{Target fiducial volume}

\subsubsection{Vertex determination}


\subsection{Data quality assurance}


\subsection{Neutral particle identification by the CDS}
%\todo[inline]{under construction, need to understand Sakuma's method }
\begin{itemize}
\item one CDH segment is fired
\item [charge veto] no CDC hits on layer 14 and on 15 in the angle of $\pm 15^\circ$ from the center of the CDH segment above
\item [isolation cuts] no hits on neighboring CDH segments
\end{itemize}

the position resolution of CDH was evaluated by charged pion tracks. 2.7 cm in phi direction and 2.1 cm in z direction.
the TOF resolution of T0-CDH is evaluated as 187 ps.


\subsection{Missing mass analysis in $d(K^-,\pi^+\pi^-n)"X"$ reaction}




\subsection{$\Sigma^{\pm}$ identification}
%kinematic fit 

\subsection{Resolution of the IM($n\pi^{+}\pi^{-}$)}

\subsection{Resolution of the momentum transfer of the reaction}

\subsection{Acceptance and Efficiency}
The total acceptance as a function of momentum transfer of the reaction (q) and invariant mass of ($n\pi^+\pi^-$) (IM), $\epsilon_{acc}(\mbox{q,IM})$ is evaluated by GEANT4 simulation.
The $n\Sigma^{\pm}\pi^{\mp}$ 3-body simulated data is generated on deuteron target with the incident beam of 1 GeV/c.
The simulated data is generated almost uniformly in (q,IM) space as shown in fig.\ref{fig:genSpSm}. 
Because by using phase space distribution, I can not get enough statistic in low mass region, I generated probability distribution from phase space simulation and put a filter in the event generator. 
Fig.\ref{fig:probSpSm} shows the probability distribution applied for GEANT4 event generator.

\begin{figure}
\includegraphics[width=0.45\linewidth]{simaccSp1.pdf}
\includegraphics[width=0.45\linewidth]{simaccSm1.pdf}\label{fig:genSpSm}
\caption{generated events distribution of $\Sigma^+$ mode (left) and $\Sigma^-$ mode (right).}
\end{figure}

\begin{figure}
\includegraphics[width=0.45\linewidth]{probSp.pdf}
\includegraphics[width=0.45\linewidth]{probSm.pdf}\label{fig:probSpSm}
\caption{probability distribution used for the GEANT4 event generator.}
\end{figure}



The total acceptance is evaluated as follows,
\[
\epsilon_{\mbox{acc}}(\mbox{q,IM}) = \frac{\mbox{Number of accepted events}}{\mbox{Number of generated events in each (q,IM) bin }}
\]
There are no cuts according to binning of (q,IM) to avoid resolution effect of reconstruction.

The $\epsilon_{\mbox{acc}}(\mbox{q,IM})$ includes the branching ratio of $\Sigma^{\pm}$. 



\subsection{$\Sigma(1385)^{-}$ analysis}
$\Sigma^0 \rightarrow \Sigma^\pm\pi^\mp$ contribution (B.R. $=11.7 \pm 1.5 \%$) to the spectra is estimated by $\Lambda \pi$ decay mode (B.R. $=87.0 \pm 1.5 \%)$.
Because we can not measure $\Sigma^0 \rightarrow \Lambda \pi^0$ decay mode directly, I have analyzed the following reaction instead,
\begin{eqnarray}
 K^-d \rightarrow \Sigma(1385)^-p \rightarrow \Lambda \pi^- p \rightarrow p\pi^-\pi^-p. \quad (87.0 \pm 1.5 \%) 
\end{eqnarray}
\subsubsection{Event selection for $K^-d \rightarrow \Sigma(1385)^-p$ analysis}
\begin{itemize}
\item \# of CDH segments fired == 3
\item \# of good CDS tracks == 3
\item $\pi^+,\pi^-$ and $p$ are found.
\end{itemize}

\subsubsection{$\Lambda$ selection}

\subsubsection{missing mass spectra of $d(K^-,p\pi^-\pi^-)$"X"}

\subsubsection{Acceptance and efficiency of $K^-d \rightarrow \Sigma(1385)^-p$}


\subsection{Systematic Error}



